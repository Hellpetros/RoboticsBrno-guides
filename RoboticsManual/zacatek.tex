 \chap Jak začít 
 
 \sec Cíl: první robot

\parskip=\smallskipamount % mezi odstavci bude mezera jako \medskip
\parindent=0pt % odstavce nebudou odsazeny zará3kou
\let\itemskip=\relax % 3ádné dal1í mezery mezi výety

Tento text je určen pro začátečníky a~mírně pokročilé v~oblasti stavby autonomních robotů -- převážně středoškoláky v~prvním ročníku, kteří se pokoušejí postavit 
svého prvního \link[ref:autonomni]{\Magenta}{autonomního }\Black  robota, nejčastěji pro nějakou soutěž\fnote{Robotiáda v~Brně: \url{http://www.robotiada.cz/}, 
Robotický den v~Praze \url{http://robotickyden.cz/} }. Protože byl sepsán na pracovišti Robotárna\fnote{\url{http://www.helceletka.cz/doku.php/pobocky/robotarna/start}} (pobočka DDM Helceletka Brno), jsou některé části určené především členům jeho kroužků. Ale většina textu je použitelná všeobecně.   

Každý robot se musí 
\begitems \style O
*vyrobit (mechanická část -- konstrukce)
*osadit elektronikou, motory a~pod. 
*naprogramovat 
\enditems

Přitom můžou nastat zhruba dvě situace: 

\begitems \style O
* Nemám žádné znalosti a~zkušenosti $\rightarrow$ doporučený postup je začít ve školním kroužku 
nebo samostatně stavět a~programovat robota z~{\it Lego mindstorms}\fnote{\url{https://www.lego.com/cs-cz/mindstorms}}. 
Je to daleko nejjednodušší a~nejrychlejší cesta, omezení jste cenou stavebnice a~jejími možnostmi. 

* Už jsem někdy něco naprogramoval nebo zapojil nebo postavil a~chci jít dál $\rightarrow$ potom je pro vás určen tento text. Jednotlivé kapitoly se věnují různým oblastem, které je postupně potřeba zvládnout (nebo jimi pověřit jiného člena týmu) s~důrazem na začátečnické problémy. 
\enditems

\secc{Týmy pro stavbu robotů}

Už bylo zmíněno, že stavba robotů zahrnuje tři propojené, ale relativně nezávislé okruhy: návrh a~výrobu mechanické konstrukce, návrh a~zapojení elektroniky a~programování. 
Proto je dobré roboty stavět v~týmech, kde se jednotliví členové zaměřují na tyto oblasti a~navzájem se doplňují. Navíc každý tým potřebuje řadu pomocných činností (nákup součástek, vyhledávání údajů na internetu a~pod.). Je dobré mít proto v~každém týmu ještě pomocníka, který podporuje ostatní a~umožňuje jim soustředit se na jejich hlavní úkoly. 

Úplně ideální potom je, když každou funkci v~týmu zastávají dva lidé, takže se mohou vzájemně zastupovat. Tým by potom měl celkem osm členů -- dva mechaniky, dva elektroniky, dva programátory a~dva pomocníky. To se ale v~praxi téměř nikdy nepodaří. Často nastane právě opačný případ, kdy tým má pouze dva nebo tři členy, kteří se o~všechny činnosti musí nějak podělit. 

V~každém případě ale platí, že je výhoda, pokud lidé v~týmu znají i~věci mimo jejich \uv{hlavní obor}, tj. když například programátor zná základy elekroniky.    


{\bf Postup návrhu robota: }
\begitems \style n
* stanovit, co by měl robot splnit -- přibližně 
* sepsat, co by měl robot splnit -- podrobně (klidně i~více stran A4), z~toho vyplyne
* zjištění, jaké senzory a~pohony robot potřebuje a~kde budou na robotovi umístěné
* návrh první konstrukce robota včetně umístění senzorů a~pohonů
* zprovoznění toho všeho -- {\bf hlavní cíl tohoto textu }
\enditems

Začátečníci obvykle tuto posloupnost nedodrží a~pak staví mechaniku pro 3 a~více robotů,
 než zjistí, že je to dost práce navíc. A~taky dost času navíc, který potom před soutěží může chybět. 




% na základě vlastních zkušeností s důkladnějším komentářem míst, kde jsem se zasekal (což bylo dost často :-) ).


\sec Co potřebujete na začátku

%\begitems \style I
{\bf Hardware: }

\begitems \style O 
* notebook nebo počítač s~operačním systémem win7 a~novějším nebo Linux (pro starší počítače např. aktuální distribuci Lubuntu, Mint)

* desku s~čipem, např. DevKit ESP32\fnote{varianta: jinou vývojovou desku, např. Arduino Uno, Arduino nano, ...  }
* vývojovou desku ALKS\fnote{\url{https://github.com/RoboticsBrno/ArduinoLearningKitStarter} \hfil \break {\localcolor \Red NUTNÁ AKTUALIZACE} 
Hotová deska vás hlavně ze začátku zbavuje nutnosti vědět, co si můžete dovolit kam připojit a~hlavně nutnosti vůbec nějaké připojování čehokoliv řešit. } 

*výhledově cokoli dalšího, co chcete tím čipem řídit (serva, ultrazvuk, senzory a~motory všeho druhu ... ) 
\enditems

\noindent {\bf Software: }
\begitems \style O 
* {\it Visual Studio code} -- viz kapitola \ref[vsc]
* {\it PlatformIO }-- viz kapitola \ref[platformio]
\enditems
Veškerý zde popisovaný a~doporučovaný software je freeware. 

\noindent {\bf Znalosti:  }
\begitems \style O
* běžná práce se soubory ve vašem operačním systému (hledání, kopírování, mazání, ... )
* základy programování C++ (viz kapitola \ref[cpp]) a~příkazy C++ pro čipy (viz kapitola \ref[cpppr]) 
*velmi se hodí schopnost porozumět textu psanému v~jednoduché angličtině) 
* základy elektroniky -- viz kapitola \ref[elektronika]
\enditems

\noindent {\bf Taky se hodí vědět, že:  } 
\begitems \style O
* veškeré \Green zelené\Black , \Blue modré \Black a~\Magenta fialové \Black odkazy a~slova jsou proklikávací (s~výjimkou barevného zvýraznění syntaxe\fnote{syntaxe -- způsob zápisu daného jazyka, barevný zápis je mnohem přehlednější} ve zdrojových textech jazyka C++)
*na konci textu je obsah 
*těsně před obsahem je rejstřík
*v žádném textu o elektronice a robotice nemůže být všechno, pokud zde něco nenajdete, použijte např. Google 
%*tento text slouží pro \uv{začátečnický rozjezd}, pravděpodobně vám proto časem přestane stačit  
\enditems
%\enditems

\sec První projekt

\nonum \notoc \secc Postup

Pokud s~programováním čipů začínáme, čekají nás tyto úkoly: 
\begitems \style n
* nainstalovat prostředí {\it Visual Studio code}   
* do VS code nainstalovat tzv. {\it PlatformIO }

* založit nový projekt 

* napsat zdrojový kód, přeložit a~dostat jej do čipu 
\enditems


\label[vsc] \secc Nainstalovat  Visual Studio code

{\it Visual Studio code} \ii Visual~Studio~code (zkráceně VS code) \ii VS~code je něco jako textový editor, speciálně navržený pro programátory čehokoliv. 
Instalujte podobně jako každý jiný program, stahujte zde: \url{https://code.visualstudio.com/}  

\label[platformio] \secc Nainstalovat PlatformIO 

{\it PlatformIO } \ii PlatformIO je ten software, který umožní program v~C++ přeložit tak, aby ho čip pochopil a~taky ho do čipu umí nahrát. 
Instalace podle návodu zde: \url{http://docs.platformio.org/en/latest/ide/vscode.html\#installation}


\secc Založit nový projekt 

Program (ne)píšete jen do jednoho souboru, ale aby vše fungovalo, 
potřebujete povícero dalších souborů, které dohromady tvoří tzv. {\bf projekt }.
\ii projekt 
Tyto soubory jsou mezi sebou hodně provázané a~navíc vázané na konkrétní místo uložení, 
takže vám po překopírování na jiný počítač pravděpodobně program nepůjde přeložit a~nebo 
nahrát do čipu. Řešení: uložit celý projekt na flešku. 
Na novém počítači založit nový projekt a~potřebné části zdrojového kódu kopírovat pomocí Ctrl+C, Ctrl+V. 

\begitems \style n
* Založte nový projekt z~ikonky \uv{domeček} -- viz \url{http://docs.platformio.org/en/latest/ide/vscode.html\#quick-start}.
* Do kolonky {\it Board} se musí vybrat správná deska. Desek je přes 400 a~jsou rozdělené do sekcí řazených podle abecedy vyznačených šedivou barvou. V~sekcích jsou potom jednotlivé desky řazené taky abecedně. 
 Pro čip ESP32 najdete sekci {\it Espressif32} a~v~ní položku {\it Espressif ESP32 Dev Module}. Kolonka {\it Framework } se potom vyplní automaticky. 
* Zbývá vybrat adresář, do kterého bude projekt uložen. Tento adresář si předem vytvořte, s~adresářem vytvářeným za pochodu má VS code kdovíproč problém. Odškrtněte zatržítko {\it Use defalut folder} a~zvolte vytvořený adresář.  
\enditems

Pro Linux Lubuntu: projekt musí být uložen na pevném disku, ne na flešce, jinak prostě nepojede, netuším proč.  

\secc Napsat zdrojový kód, přeložit a~dostat jej do čipu 

\midinsert
\picw=\hsize \centerline{\inspic soubory/rozlozeni2.jpg }
\nobreak\medskip
\label[vsc_rozlozeni] \caption/f Rozložení oken v~programu {\it VS code} 
\endinsert

Obrázek \ref[vsc_rozlozeni] na straně~\pgref[vsc_rozlozeni]  ukazuje rozložení oken v~rámci projektu. Hlavní okno  rozdělte na dvě části pro zobrazení dvou upravovaných souborů pomocí ikony v~kroužku. Začínáme v~okně \dest[ref:explorer] {\it Explorer}, \ii explorer kde je umístěna adresářová struktura projektu\fnote{Pokud toto okno není vidět, zobrazíte ho v~menu {\it View} položka {\it Explorer} }. Otevřete soubory {\it platformio.ini} a~v~adresáři {\it src} soubor {\it main.cpp}. 
 
Pro pohodlnou práci s~deskou ALKS byla \ii LearningKit napsaná tzv. knihovna {\it LearningKit}. Aby fungovala, musí být do souboru {\it platformio.ini} dopsán řádek 
{\tt lib\_deps = 1745} (bez mezery na začátku řádku)
a~do záhlaví souboru {\it main.cpp} doplňte {\tt \#include "LearningKit.h"}.  
Dále dopište do souboru {\it main.cpp} kód, který bliká červenou LED. Vše je vidět na obrázku \ref[vsc_rozlozeni]. Celý zdrojový kód tohoto prvního programu (obsah souboru {\it main.cpp}) je uveden v~kapitole \ref[cpppr1].
 
Teď budou potřeba další dvě části VS code: \iid terminál (okno vpravo dole) a~stavový řádek ( \iid Status~bar -- proužek pod terminálem). Na stavovém řádku klikněte na ikonu šipky\fnote{totéž provede Ctrl+Alt+U} (pátá zprava) a~{\it platformIO} se pokusí váš program přeložit a~nahrát do čipu. Pokud chcete program pouze přeložit, klikněte na ikonu zatržítko\fnote{totéž provede Ctrl+Alt+B} hned vedle. 

Při prvním pokusu nahrát program do čipu na linuxu má {\it platformIO} problém, 
že nenajde USB spojení na desku s~čipem a~vyžaduje ho doistalovat. 
Zpráva\fnote{\tt Warning! Please install `99-platformio-udev.rules`} 
se objeví se v~terminálu včetně nápovědy,\fnote{\url{https://raw.githubusercontent.com/platformio/platformio/develop/scripts/99-platformio-udev.rules}}
 jak to udělat. Nápověda je ale tak podrobná, že to středně poučený linuxový laik s~pomocí internetu zvládne. Při všech dalších překladech už to nebude problém.  

Další programy budou uvedeny v~kapitole \ref[cpppr].
