 
 
\section{\LaTeX} 

\subsection{Proč používat \LaTeX} 

Tento text je psán v~sázecím systému \index{LaTeX} \LaTeX . 
Jeho silná stránka je především matematická sazba 
(bohužel nevyužijeme) a~snadné zpracování obsahu, rejstříků, seznamů obrázků a~tabulek a~podobně, což dramaticky urychluje přípravu dokumentu. 
Čas, který vložíte do nastavení a~učení se systému, se vrátí v~rychlosti práce
 $\rightarrow$ jedná se o řešení vhodné pro delší dokumenty, např. pro soutěž SOČ nebo dlouhodobou maturitní práci. 
% Existují různé nadstavby a~rozšíření \TeX{}u, nejznámější je \LaTeX{}.
 
Návody pro \LaTeX lze najít v 
\url{http://mirrors.nic.cz/tex-archive/info/czech/\|latex-pro-pragmatiky/latex-pro-pragmatiky.pdf} %todo odkaz nejede
\emergencystretch=2cm a na internetu.
Hodně vám také může pomoct zdrojový text této dokumentace. 

Příkazy \LaTeX{}u podobně jako C++ a~systémy typu Linux {\bf rozlišují velká a~malá písmena}. 
 

\subsection{Editory}

Ve WinXP používám PSpad, v~linuxu TeXstudio. 
Oba editory umí zavolat překlad do pdf pomocí klávesové zkratky, zobrazit výsledný pdf a~barevné zvýraznění syntaxe. 
PSpad se dá rozjet i~pod linuxem pomocí prostředí Wine, ale nepodařilo se mi rozchodit volání překladu. 


\section{Github}

\subsection{Proč a jak se používá}

Pokud více lidí pracuje průběžně na stejném projektu, musejí nějak sdílet výsledky svojí práce. K tomu se používají 
tzv. repozitáře a Github\footnote{\url{www.github.com}} je v současnoti jeden z nejpoužívanějších webů pro tvorbu a správu repozitářů. 
%todo doplnit vysvětlení, co všechno github umí 


Postup práce je následující: na webu github.com si vytvořím repozitář (viz \ref{instal_github}) . Stáhnu k sobě na počítač aktuální verzi repozitáře, upravím, co potřebuji a upravené soubory nahraju zpět na server. 

Repozitář, ve kterém je i tato dokumentace, je na adrese 
\url{https://github.com/RoboticsBrno/RobotikaBrno-guides/tree/RoboticsManual}.  


\subsection{Instalace githubu a stažení repozitáře} \label{instal_github}

\begin{enumerate}
	\item Na github.com vytvořím účet a přihlásím se. Vpravo můžu založit nový repozitář (new repository) nebo se můžu přepnout do už existujících repozitářů. 
	
	\item Na svém počítači si naunstaluji github. V linuxu napíšu do terminálu: {\tt sudo apt-get install git}.  
	
	\item Vytvořím si lokální adresář a stáhnu do něj repozitář. V linuxu napíšu do terminálu: {\tt git clone <cestu k repozitáři> }. 
	Cestu zjistím tak, že se na webu přepnu do repozitáře a kliknu na tlačítko \uv{Clone or download}. Ose bjeví 
\end{enumerate}


 
\section{Další software}

\subsection{Proficad}

Proficad\footnote{Instalační soubor seženete v~kroužku nebo na webu.} 
je software určený původně pro snadné a~rychlé kreslení elektronických schémat a~v~této oblasti je vynikající. 
Lze jej použít i~jako jednoduchý vektorový editor obrázků. \index{Proficad}

SPŠ Sokolská zakoupila plnou multilicenci pro Proficad, takže studenti i~učitelé jej mohou používat bez omezení. 

Ovládání programu je velice intuitivní a~nápovědu prakticky nepotřebujete -- s~jedinou výjimkou, a~tou je nastavení rastru. 
Po instalaci je rastr zobrazení automaticky nastaven na 2 mm. To znamená, že součástky 
můžete umisťovat například 10 mm nebo 12 mm od kraje, ale nic mezi tím. 
Většinou se to hodí -- součástky máte na schématu pěkně zarovnané -- ale 
někdy je prostě potřeba rastr například vypnout neboli nastavit na nulu. 
Nastavení rastru je schované zde:  {\it soubor/nastavení/dokument/obsah/rastr}.

\subsection{Prohlížení Pdf}

Ve WinXP se osvědčil Foxit Reader verze 2.3, v~linuxu Evince.  
%todo doplnit odkazy a další software
