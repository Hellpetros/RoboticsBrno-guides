\documentclass[12pt]{article} % article - typ/šablona dokumentu
\usepackage[czech]{babel} % nastavuje české popisky např. u obsahu, referencí, tabulek, obázků 
\usepackage[utf8]{inputenc} % použito UTF8 kvůli češtině (zvládá prakticky všechny jazyky na světě)

%\usepackage{indentfirst} % odsazuje první odstavec v kapitole

\usepackage{color} % balíček pro obarvování textů
% \color{blue}
\usepackage{xcolor}  % zapne možnost používání barev, mj. pro \definecolor

\usepackage{hyperref} % balíček pro hypertextové odkazy
% \url{www.odkaz.cz}
% \href{http://www.odkaz.cz}{Text který bude jako odkaz}
%\hyperlink{label}{proklikávací_text} - odkaz na text 
% \hypertarget{label}{cíl_odkazu} - cíl odkazu  


\definecolor{mygreen}{RGB}{0,150,0} % nastavení barvy odkazů 
\definecolor{myblue}{RGB}{0,0,200} 
\hypersetup{colorlinks=true, linkcolor=myblue, urlcolor=mygreen, citecolor=blue, anchorcolor = magenta, linktocpage = true, frenchlinks } % nastavení barvy odkazů


\usepackage{makeidx} % slouží pro vytváření indexů/rejstříků
\makeindex % zapíná vytváření/ukládání indexů
% \index{key} % ukládá index
% \printindex % tiskné indexy/rejstřík

\usepackage{graphicx} % pro vkládání obrázků a příkaz "\includegraphics"
%% samotné vložení obrázku
%\begin{figure}
%	\includegraphics[width=\textwidth]{../img/when_use_latex.png}
%	\caption{Kdy se vyplatí použít \LaTeX} % popis, který se zobrazí pod obrázkem
%	\label{moje_navesti} %identifikuje objekt, který lze pak referencovat
%\end{figure}
%% ---------------------------
  
  
  
%opening
%\title{}
%\author{}

\begin{document}

%% makra pro PlainTex?
% \input makra1.tex

%% určitě je pro to v LaTeXu alternativa -> OK, prosím o její nalezení 
% \draft % tiskne na poslední stranu seznam komentářů, co se má dodělat , viz opmac.tex ř. 889


\input prvni_s.tex
\input zacatek.tex
\input zac_pojmy
%\input cpp.tex
%\input cpp2.tex
%\input cpppr.tex
%\input elektronika.tex
% \input bity_bajty.tex
%\input software.tex
 

%\chap Ostatní

%\input mechanika.tex
 
\section{Řešení některých problémů}

\subsection{Chyba při uploadu programu do desky Arduino nano:}

% \begtt Please specify 'upload_port' environment or use global '--upload-port' option \endtt

Řešení: pomohlo spustit příkaz {\tt sudo service udev restart} a znovu spustit VS code a odpojit a připojit desku.  

% \input prilohy

% \vfil \break \nonum \sec Rejstřík 

% \begmulti 3 
% \makeindex 
% \endmulti

% \vfil \break \nonum \sec Obsah %\notoc

% \maketoc


% \makerfc

% \bye
 
  
% "-translate-file=cp1250cs" "%Name%.tex"
% \input todo.tex


\printindex % vytiske rejstřík


\tableofcontents % zobrazí obsah

\end{document}

