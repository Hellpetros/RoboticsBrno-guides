\chap Programovací jazyk C++

\sec Základy syntaxe v~jazyce C

\secc Základní pojmy

Upozornění: V~jazyce C++ se {\bf rozlišují velká a~malá písmena}.  

\begitems \style o

* {\bf příkaz} \ii{příkaz} -- povel pro procesor, říká procesoru, co má dělat.
Příkazy probíhají postupně, nejprve se udělá jeden a~pak další, který je na řadě
Za příkazy dáváme středník (;).

* {\bf syntaxe} \ii{syntaxe} -- \uv{pravopis} programovacího jazyka, říká, jak psát příkazy programovacího jazyka tak, aby nám počítač rozuměl 

* {\bf preprocesor} \ii{preprocesor} -- před samotným překladem programu do {\it  \char`\*.hex} souboru proběhnou tzv. direktivy preprocesoru.\fnote{něco jako příkazy, podrobně viz např. Kadlec: Programování v~jazyce C, str. 18 } \ii direktiva~preprocesoru \nl 
Příklady: 

 { \tt \#include $<>$ } -- \ii include pro vkládání hlavičkových souborů neboli {\bf knihoven} \ii knihovna \ii hlavičkový/soubor,@

 { \tt \#include "" } -- pro vkládání vlastních hlavičkových souborů  
 
 {\tt \#define KONSTANTA HODNOTA\_KONSTANTY} -- \ii define pro definici konstanty, všude kde se v~kódu vyskytne text {\tt KONSTANTA} bude tento text nahrazen  {\tt HODNOTA\_KONSTANTY}

* {\bf proměnná} \ii{proměnná}  -- je místo v~paměti, kde jsou uložena nějaké data

* {\bf datový typ} \dest[ref:datovy_typ]
 \ii datový/typ,@ -- říká nám, v~jaké podobě jsou data v~proměnné uložena. V~proměnné jsou pouze jedničky a~nuly, Pomocí datového typu mikrokontrolér pozná, jestli ta data jsou čísla, nebo znak, v~jakém rozsahu jsou čísla, atd...
používané datové typy:   \nl
{\tt uint8\_t} -- celá čísla od 0 do 255  \rfc{zkontrolovat typy dat} \nl
{\tt int8\_t} -- celá čísla od -127 do 127 \nl
{\tt uint16\_t} --  celá čísla od 0 do 65535  \nl
{\tt int16\_t} -- celá čísla od -32767 do 32767 \nl
{\tt uint32\_t} -- celá čísla od 0 do 4294967296 \nl
{\tt int32\_t} -- celá čísla od -2147483647 do 2147483647 \nl

* {\bf operandy} \ii operand -- jsou čísla nebo výrazy, které \uv{vstupují do operace} 
* {\bf operátory} \ii operátor -- nám říkají, jaké operace provedeme s~operandy (co s~nimi uděláme) 

Příklad: {\tt 5 + 2}, přitom {\tt 5} a~{\tt 2} jsou operandy a~znaménko plus je operátor, který nám říká, že čísla chceme sečíst (provést operaci sečítání)


* {\bf poznámka} \ii poznámka -- cokoliv se objeví v~kódu mezi znaky {\tt /\char`\*} a~{\tt \char`\*/}, tak je poznámka a~bude před překladem odstraněno z~kódu. 

Pokud chceme zapoznámkovat pouze jeden řádek, použijeme {\tt //}

Příklady:
\hisyntax{C} \begtt 
/* Vše co napsáno zde
je poznámka a nebude překládáno
*/
\endtt

\hisyntax{C} \begtt 
// Toto je poznámka, která může popsat funkci nějakého úseku kódu
\endtt

\enditems

%%%%%%%%%%%%%%%%%%%%%%%%%%%%%%%%%%%%%%%%%%%%%%%%%%%%%%%%%


\secc{Výrazy a~operátory}

Výraz \ii výraz je něco, co nabývá nějaké hodnoty, např.: { \tt  5 + 3} je výraz, který nabývá hodnoty { \tt  8}. { \tt  5 > 3} je výraz nabývající hodnoty { \tt  true} neboli pravda, { \tt  x <= 10} je výraz který je pravdivý, pokud proměnná { \tt  x} (kterou musíme mít deklarovanou) menší nebo rovna číslu { \tt  10}. 

%\begitems \style o

{ \bf Aritmetické operátory} 

{ \tt  +} sčítání 

{ \tt  -} odčítaní 

{ \tt  \char`\*} násobení 

{ \tt  /} dělení 

{ \bf Operátory inkrementace a~dekrementace}

pokud chceme zvýšit hodnotu proměnné o jedničku, napíšeme 
{ \tt  ++název\_proměnné};
pokud chceme snížit hodnotu proměnné o jedničku, napíšeme { \tt  --název\_proměnné};

Příklad:

\hisyntax{C} \begtt 
uint8_t i = 5;  // Vytvoření proměnné i a dosazení hodnoty 5 do ní.  
uint8_t j = 42; // Vytvoření proměnné j a dosazení hodnoty 42 do ní. 

++i;  // Hodnota i se zvýší o jedničku na číslo 6.
--j;  // Hodnota j se sníží o jedničku na číslo 41.
\endtt


%\subsubsection{Přiřazovací operátor}
%= 

{ \bf Logické výrazy a~operátory} 
 
{ \tt  ==} porovnání - rovnost 
 
{ \tt  !=} porovnání - nerovnost 

{ \tt  \&\&} logický součin 
 
{ \tt  ||} logický součet 

{ \tt  ! } negace 
 
{ \tt  $<$ } menší než 

{ \tt  $<=$} menší nebo rovno 

{ \tt  $>$ } větší než 

{ \tt  $>=$} větší nebo rovno 

%\enditems

{\bf Logický výraz} \ii logický/výraz,@ může nabývat pouze dvou hodnot -- pravda (true) 
a~nepravda (false).  Obvykle je nepravda reprezentovaná nulou a~pravda každým nenulovým číslem. Logické výrazy se používají v~podmínkách 
(viz např. kap. \ref[if])

{\bf Logické operace} \ii logická/operace,@ se používají při vyhodnocování logických výrazů.    


Příklady: 
l
{ \tt  a~== b} výraz je pravdivý, pokud se { \tt  a} rovná { \tt  b} 
l
{ \tt  a~!= b} výraz je pravdivý, pokud se { \tt  a} nerovná { \tt  b}. 

Lze to napsat i~konkrétněji, např.: { \tt  a~!= 5}, výraz je pravda, pokud se { \tt  a} nerovná { \tt  5}, pokud se rovná, výsledkem výrazu je nepravda(false).

{ \tt  b > c} výraz je pravdivý, pokud je { \tt  b} větší než { \tt  c} 

{ \tt  b >= d} výraz je pravdivý, pokud je { \tt  b} větší nebo rovno { \tt  d} 

{ \tt  !(a == b)} výraz { \tt  (a~== b)} je negován znaménkem { \tt  !}, to znamená, že výraz je pravdivý, pokud se výraz { \tt  a} nerovná výrazu { \tt  b}

Pozor: negace výrazu { \tt  (a~> b)}, tj. { \tt  !(a > b)} není to samé jako výraz { \tt  (a~< b)}, ale správně je to { \tt  (a~<= b)} \fnote{Platí zde určité zákonitosti viz De Morganovy zákony.} 


Logický součin {\tt  \&\&} se používá, pokud budeme potřebovat spojit dva nebo více výrazů dohromady, např.: {\tt  (a~$>$ b)\&\&(c $==$ d)}, výsledkem tohoto výrazu bude \ii logický/součin,@ výrazů v~závorkách. Pro logický součin platí, že je pravda pokud oba výrazy jsou pravdivé, jinak je výsledek nepravda. 

Tedy, zde bude výsledkem pravda, pouze pokud bude { \tt  a} větší než { \tt  b} a~zároveň bude platit, že { \tt  c} se rovná { \tt  d}.

Logický součin použijeme, pokud musí všechny výrazy být pravda.

Logický součet \ii logický/součet,@ { \tt  ||} je pravdivý, pokud alespoň jeden výraz je pravdivý. 

Např.: { \tt  (e $<=$ f)||(g $!=$ 3)} výraz bude pravda, pokud bude platit, že { \tt  e} je menší nebo roven { \tt  f}, nebo bude platit, že { \tt  g} se nerovná { \tt  3}, anebo klidně budou platit oba výrazy.

