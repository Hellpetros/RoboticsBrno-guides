\sec Mechanická konstrukce robota -- doporučení
\label[konstrukce] \ii konstrukce~robota \ii robot/konstrukce  

Roboty rozlišujeme podle způsobu ovládání a~podle velikosti. 

\secc Ovládání robotů

\ii autonomní/robot,@ Každý robot může být buď řízený nebo autonomní.\dest[ref:autonomni] {\bf Autonomní }znamená, že je naprogramovaný a~během soutěže nebo prezentace se pohybuje samostatně.
  

Roboty můžeme řídit po kabelu nebo bezdrátově, například přes 
\link[ref:bluetooth]{\Magenta}{bluetooth}\Black. 
 
\secc Velikost robotů
 
Na škole a~v~DDM stavíme a~programujeme hlavně roboty dvojího typu :
\uv{střední} a~\uv{velké}. 

  Jako střední používáme roboty {\it yunimin} a~{\it 3pi} - jsou v~zásadě hotové
a~\uv{pouze} se programují, jsou určené pro soutěže typu sledování čáry (Robotický den v~Praze, Istrobot), minisumo a~další. Pro střední roboty neexistuje soutěž, kde by se ovládali po kabelu, musí proto být autonomní.    
 
 Velké roboty (typicky pro soutěž Eurobot a~Robotický den) si stavíme sami. Mohou být dálkově řízené i~autonomní. Začátečníkům se silně doporučuje začít s~dálkově řízeným robotem a~nebo se přidat k~týmu zkušenějších konstruktérů.   
 
 \secc{Konstrukce velkých robotů}
 
Celého robota nejprve navrhneme ve vhodném \iid CAD programu: Solidworks\fnote{\url{http://www.solidworks.com}}, onshape\fnote{\url{ doplnit}}, \rfc{odkaz na onshape}   
 \rfc{Fusion?}. \ii Solidworks \ii onshape 
 Pokud jako základní materiál zvolíme překližku nebo například plexisklo (kombinované s~díly z~lega), můžeme díly vymodelované v~CADu nechat vyřezat na laseru na pracovišti Fablab.\fnote{\url{https://www.fablabbrno.cz}} To dramaticky urychluje práci.    
 Pro konstrukci prototypů se také hodí měkčené PVC.  

 Při stavbě velkých robotů se držíme osvědčeného schématu: 

  Podvozek tvoří základní deska z~překližky nebo rám z~hliníkových profilů typu \uv{L} tvaru obdélníku, osmiúhelníku nebo kruhu.  Na podvozku jsou  připevněné dva motory z~akušroubováků nebo z~akuvrtaček včetně převodovek, které se koupí hotové. Jiné typy motorů jsou buď pomalé nebo slabé. V~žádném případě se nepokoušejte koupit pouze motor (např. protože je levnější) a~vyrábět si převody sami.  
 
 Dále jsou zde dvě kola, každé připojené ke svému motoru a~jedna nebo více podpěr podvozku, obvykle z~kartáčku na zuby. Motory s~koly lze umístit doprostřed nebo dozadu, podle toho, co má robot na hřišti dělat. 
 
 Na podvozek se umisťuje konstrukce z~hliníkových tyčí, profilů nebo z~merkuru, s~pomocí které robot plní svoje úkoly. Méně tuhé materiály se neosvědčily.    
 

  
