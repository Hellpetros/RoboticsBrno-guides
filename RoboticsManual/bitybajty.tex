\section{Desítková soustava a~dvojková soustava }

V~počítači jsou všechna data uložena pouze v~podobě jedniček a~nul. 
A~proto se nám bude hodit, když se budeme aspoň trochu orientovat v~převodech mezi dvojkovou a~desítkovou soustavou.

\subsection{Desítková soustava}
V~\B{desítkové soustavě}\I{desítková soustava}\I{soustava!desítková} je základem číslo deset. 

  Příklad: číslo 156 si můžeme rozložit jako \verb@1*10^2 + 5*10^1 + 6*10^0@, přitom
\verb@10^0=1, 10^1=10, 10^2=10*10=100, 10^3=10*10*10=1000@ atd ...,
takže naše číslo potom vypadá takto: \verb@ 1*100 + 5*10 + 6*1 = 156@.

\subsection{Dvojková soustava}
Ve \B{dvojkové soustavě}\I{dvojková soustava}\I{soustava!dvojková} je základem číslo dva. 

  Příklad: číslo \verb@11001010@ rozložíme na \\
\verb@1*2^7 + 1*2^6 + 0*2^5 + 0*2^4 + 1*2^3 + 0*2^2 + 1*2^1 + 0*2^0@, přitom \\
\verb@2^0=1, 2^1=2, 2^2=2*2=4, 2^3=2*2*2=8, 2^4=2*2*2*2=16, 2^5=32@, \\
 atd..., proto naše číslo má hodnotu: \\
\verb@1*128 + 1*64 + 0*32 + 0*16 + 1*8 + 0*4 + 1*2 + 0*1 = 202@.   

\subsection{Převod z~desítkové soustavy do dvojkové}
Jak na to, když chceme převádět číslo z~desítkové soustavy do dvojkové?

  Například převedeme číslo \verb@97@. \\ 
\noindent Nejprve je potřeba, abychom znali násobky čísla dva: \\ 
\verb@2, 4, 8, 16, 32, 64, 128, 256, 512, 1024, 2048@, atd... \\
Napíšeme jedničku. \\
\verb@1@ \\ 
Najdeme nejbližší nižší násobek dvou k~našemu číslu, v~našem případě je nejblíže \verb@64@.  
Odečteme si od \verb@97@ číslo \verb@64@, \verb@97-64=33@. Odečítali jsme číslo \verb@64@, 
další v~pořadí je číslo \verb@32@, je \verb@32@ obsaženo v~\verb@33@? Ano, proto napíšeme další jedničku. \\ 
\verb@11@ \\ 
Dále \verb@33-32=1@, další číslo je \verb@16@, je \verb@16@ obsaženo v~\verb@1@? Ne, proto napíšeme nulu.
\verb@110@ \\ 
Další číslo je \verb@8@, také není obsaženo v~\verb@1@, takže napíšeme nulu. \\ 
\verb@1100@ \\ 
Čísla \verb@4@ a~\verb@2@ také nejsou obsažena v~čísle \verb@1@, proto napíšeme nuly. \\
\verb@110000@ \\
A~nakonec, je číslo jedna obsaženo v~čísle jedna? Ano, napíšeme jedničku.
Náš binární zápis čísla \verb@97@ vypadá takto:\verb@1100001@. 
Čísla v~binární soustavě by měla mít počet cifer rovno 8, 16, 32, 
podle toho s~kolika bitovými čísly počítáme. V~našem případě musíme doplnit jednu nulu. 
Pokud bychom počítali s~16 bitovými čísly doplníme o 8 nul více. Nuly doplňujeme zleva.
 Zapamatujeme si to snadno v~desítkové soustavě se číslo \verb@128@ také nezmění pokud před něj dáme nuly,
  \verb@000128@. Nuly navíc nebudeme psát v~desítkové soustavě, ale v~dvojkové nám to pomůže se lépe orientovat. 
  Takže to bude vypadat: \verb@01100001@ 
No a~to ještě není všechno. Aby mělo číslo správný tvar, musí mít před číslem uvedeno \verb@0b@(nula a~malé B),
 tím říkáme v~jazyce C, že číslo je v~(binární) dvojkové soustavě. 

Pozor: Čísla \verb@101@ a~\verb@0b101@ nejsou to samé! Je to jednoduché, 
pokud \verb@0b@ neuvedeme bude překladač číslo brát jakože je v~desítkové soustavě. Tj. \verb@101 =>@ číslo sto jedna, \verb@0b101 =>@ číslo pět.

  \B{Další příklad:}\\
Převedeme číslo \verb@237@. \\ 
Nejbližší nižší číslo je \verb@128@. \\ 
 \verb@1@  \\
\verb@237-128=109@ je číslo \verb@64@ obsaženo v~čísle \verb@109@? ANO, další jednička. \\
\verb@11@ \\ 
\verb@109-64=45@ je číslo \verb@32@ obsaženo v~\verb@45@? ANO, jednička. \\
\verb@111@  \\ 
\verb@45-32=13@ je \verb@16@ obsaženo v~čísle \verb@13@? NE, nula. \\ 
\verb@1110@  \\ 
je číslo \verb@8@ obsaženo v~\verb@13@? ANO, jednička. \\
\verb@11101@ \\ 
\verb@13-8=5@ je \verb@4@ obsaženo v~\verb@5@? ANO, jednička. \\
\verb@111011@ \\ 
\verb@5-4=1@ je  obsažena v~\verb@1@? NE, nula. \\
\verb@1110110@ \\
je \verb@1@ obsažena v~\verb@1@? ANO, jednička. \\
\verb@11101101@
Binární zápis čísla \verb@237@ vypadá takto:\verb@11101101@.


\subsection{Převod z~dvojkové soustavy do desítkové}

Vezmeme si například binární číslo \verb@1011@. Očíslujeme si pozice zprava doleva od nuly, tj.:
\begin{verbatim}
1 0 1 1
3 2 1 0
\end{verbatim}
Pohybujeme se v~dvojkové soustavě, tj. základ bude \verb@2@. Vezmeme nultou číslici a~vynásobíme s~ní číslo \verb@2^0@. 
Dostáváme \verb@1*2^0 = 1@.
Dále první číslici vynásobíme číslem \verb@2^1@, tj. \verb@1*2^1 = 2@.
Dále druhou číslici vynásobíme číslem \verb@2^2@, tj. \verb@0*2^2 = 0@.
A~to samé provedeme s~číslicí na třetí pozici: \verb@1*2^3 = 8@.
A~nyní čísla sečteme \verb@8 + 0 + 2 + 1 = 11@, takže binární číslo \verb@1011@ převedené do desítkové soustavy je \verb@11@.

Příklad 2: Převeďte binární číslo \verb@11101001@ do desítkové soustavy.\\
\verb@1*2^0 + 0*2^1 + 0*2^2 + 1*2^3 + 0*2^4 + 1*2^5 + 1*2^6 + 1*2^7@
To je: \\
\verb@1 + 0 + 0 + 8 + 0 + 32 + 64 + 128 = 233@
Binární číslo \verb@11101001@ převedeno do desítkové soustavy je \verb@233@.

Příklad 3: Převeďte binární číslo \verb@00010101@. \\
\verb@1*2^0 + 0*2^1 + 1*2^2 + 0*2^3 + 1*2^4 + 0*2^5 + 0*2^6 + 0*2^7@, 
samozřejmě členy \verb@2^x@, které násobíme nulou budou nulové a~tudíž je můžeme beztrestně vynechat, tj. \\
\verb@1*2^0 + 1*2^2 + 1*2^4 = 1 + 4 + 16 = 21@
Binární číslo \verb@00010101@ má v~desítkové soustavě hodnotu \verb@21@. 

\subsection{Bitové výrazy a~práce s~nimi}

Bitové operace se od logických operací (viz \ref{logika1}) liší tím, 
že se neaplikují na číslo jako celek, ale bit po bitu, viz níže. 

My budeme potřebovat následující bitové operace: bitový součet, bitový součin, bitovou negaci, bitový posun doleva.

\subsubsection{Bitové operátory}
\verb@|@ bitový součet \\
\verb@&@ bitový součin \\
\verb@~@ bitová negace \\
\verb@^@ bitový exluzivní součet \\
\verb@<<@ bitový posuv doleva \\
\verb@>>@ bitový posuv doprava \\ 

\subsubsection{Bitový součet} Platí, že \verb@0|0=0, 0|1=1, 1|0=1, 1|1=1@, sčítají se vždy pouze bity, které leží pod sebou. 
Zkrátka pokud je aspoň jeden bit jedna, výsledek je také jedna. 

Příklad: máme dvě binární čísla \verb@00000100@ a~\verb@01100001@ a~chceme je bitově sečíst. 

  Když sečteme nulté bity zadaných čísel, dostamene 0|1=1. Pak provedeme bitový součet i~pro první, druhé a~další bity.
Zapíšeme:
\verb@(0b00000100 | 0b01100001);@
Nebo pokud budeme mít čísla v~proměnných:
\begin{verbatim}
uint8_t a~= 0b00000100;
uint8_t b = 0b01100001;
uint8_t c;
c = (a|b);
\end{verbatim}
Do proměnné \verb@c@ se zapíše výsledná hodnota.

\begin{verbatim}
00000100 - první číslo
01100001 - druhé číslo
--------
01100101 - výsledek po bitovém logickém součtu 
\end{verbatim}

Výsledek: \verb@01100101@.

\subsubsection{Bitový součin} 
Platí, že \verb@0&0=0, 0&1=0, 1&0=0, 1&1=1@, tady platí, že výsledek je jedna, pokud oba bity jsou jedna.
Příklad: bitový součin čísel \verb@00110011@ a~\verb@01000001@
Zapíšeme:
\verb@(0b00110011 & 0b01000001);@
Nebo pokud budeme mít čísla v~proměnných:
\begin{verbatim}
uint8_t c = 0b00110011;
uint8_t d = 0b01000001;
uint8_t e;
e = (c&d);
\end{verbatim}
Výsledná hodnota se zapíše do proměnné \verb@e@.
\begin{verbatim}
00110011 - první číslo
01000001 - druhé číslo
--------
00000001 - výsledek bitového logického součinu
\end{verbatim}

\subsubsection{Bitová negace}
Tam, kde byla jednička, bude nula, a~kde byla nula, bude jednička.
Příklad: znegujte číslo \verb@00110101@
Zapíšeme:
\verb@~(0b00110101);@
Nebo pokud budeme mít číslo v~proměnné:
\begin{verbatim}
uint8_t a~= 0b00110101;
uint8_t b;
b = ~a;
\end{verbatim}
Do proměnné \verb@b@ se zapíše výsledná hodnota.
\begin{verbatim}
00110101 - číslo, které bude logicky znegováno
--------
11001010 - výsledek
\end{verbatim}


\subsubsection{Bitový exklusivní součet}
Výsledkem exklusivního součinu je pravda, pokud se hodnoty liší, tj. nejsou stejné. Takže platí \verb@0^0=0, 0^1=1, 1^0=1, 1^1=0@

Zapíšeme:
\verb@(0b00110011 ^ 0b01000001);@
Nebo pokud budeme mít čísla v~proměnných:
\begin{verbatim}
uint8_t c = 0b00110011;
uint8_t d = 0b01000001;
uint8_t e;
e = (c^d);
\end{verbatim}
Výsledná hodnota se zapíše do proměnné \verb@e@.
\begin{verbatim}
00110011 - první číslo
01000001 - druhé číslo
--------
01110010 - výsledek bitového exklusivního součtu
\end{verbatim}

\subsubsection{Bitový posun doleva} 
Vezmeme celé osmibitové číslo a~posuneme ho o určitý počet míst doleva. 
Přitom čísla vlevo zmizí a~na na uprázdněné místo vpravo přijdou vždy nuly.
Zapíšeme:
\verb@(0b01000101 << 3);@
Nebo pokud budeme mít číslo v~proměnné:
\begin{verbatim}
uint8_t c = 0b01000101;
uint8_t d;
d = (c << 3);
\end{verbatim}
Do proměnné \verb@d@ se přiřadí výsledek operace.
Pokud například už nebudeme potřebovat původní hodnotu proměnné \verb@c@, lze to napsat i~takto:
\begin{verbatim}
uint8_t c = 0b01000101;
c = (c << 3);
\end{verbatim}
Hodnota v~proměnné \verb@c@ se nejprve posune o tři pozice a~pak se výsledná pozice zapíše na stejnou pozici v~paměti mikrokontroléru.

Příklad: proveďte bitový posun doleva o tři s~číslem \verb@00000101@

\begin{verbatim}
01000101
--------
10001010 - posun o jedno doleva 
00010100 - znovu o jedno doleva (celkem o dvě místa)
00101000 - celkový posun o tři bity doleva 
\end{verbatim}

\subsubsection{Bitový posun doprava}
Bitový posun doprava je podobný jako operace bitový posun doleva, liší se jen ve směru posunu, tj. posouvá se na opačnou stranu.
Příklad 1: Proveďte bitový posun doprava o tři pozice s~binárním číslem \verb@00000101@. Zapíšeme:
\verb@(0b00000101 >> 3);@
Výsledkem bude toto:
\begin{verbatim}
00000101
--------
00000010 - posun o jednu pozici doprava
00000001 - posun o druhou pozici
00000000 - posun o třetí pozici
\end{verbatim}

Příklad 2: Proveďte bitový posun o dvě pozice doprava a~potom o tři doleva s~binárním číslem \verb@11110111@. Zapíšeme:
\verb@((0b11110111 >> 2) << 3);@
Výsledkem bude toto:
\begin{verbatim}
11110111
--------
00111101 - posun o dvě pozice doprava
11101000 - posun o tři pozice doleva, toto je výsledek.
\end{verbatim}

Příklad 3: Ale co když zadám číslo do proměnné v~desítkové soustavě, například \verb@189@ a~provedu bitový posun o dvě pozice doprava? 
Co dostanu? Dostanu snad \verb@1@? Ne, i~když jsme číslo zadali v~desítkové soustavě, v~mikrokontroléru je 
stejně uloženo jako jedničky a~nuly, takže číslo musíme nejprve převést do dvojkové soustavy.
\verb@189 = 0b10111101@
A~teď posun o dvě pozice doprava:

\verb@00101111@ a~to je číslo \verb@47@.
Takže \verb@47@ je výsledek bitového posunu čísla \verb@189@ o dvě pozice doprava.

Zapíšeme takto:
\begin{verbatim}
uint8_t a~= 189;
uint8_t b;
b = (a~>> 2);
\end{verbatim}
No a~výsledek(tj. číslo \verb@47@ ) se uloží do proměnné \verb@b@.
