\documentclass[12pt]{article} % article - typ/šablona dokumentu
\usepackage[czech]{babel} % nastavuje české popisky např. u obsahu, referencí, tabulek, obázků 
\usepackage[utf8]{inputenc} % použito UTF8 kvůli češtině (zvládá prakticky všechny jazyky na světě)

\usepackage{color} % balíček pro obarvování textů
% \color{blue}

\usepackage{hyperref} % balíček pro hypertextové odkazy
% \url{www.odkaz.cz}
% \href{http://www.odkaz.cz}{Text který bude jako odkaz}

\usepackage{makeidx} % slouží pro vytváření indexů/rejstříků
\makeindex % zapíná vytváření/ukládání indexů
% \index{key} % ukládá index
% \printindex % tiskné indexy/rejstřík

\usepackage{graphicx} % pro vkládání obrázků a příkaz "\includegraphics"
%% samotné vložení obrázku
%\begin{figure}
%	\includegraphics[width=\textwidth]{../img/when_use_latex.png}
%	\caption{Kdy se vyplatí použít \LaTeX} % popis, který se zobrazí pod obrázkem
%	\label{moje_navesti} %identifikuje objekt, který lze pak referencovat
%\end{figure}
%% ---------------------------

%opening
\title{}
\author{}

\begin{document}


%% v LaTeXu lze myslím taky nějak udělat + nevím jestli to balíček hyperref nedělá automaticky
% \hyperlinks{\Blue}{\Green} % Pokud napíšete na začátek dokumentu \hyperlinks{<color in>}{<color out>}, pak se v dokumentu při výstupu do PDF stanou klikacími:

% \typosize[10/12] %úvodní nastavení velikosti fontů - potřeba pro \typoscale
% \outlines{1} % přehledový obsah v levé záložce PDF prohlížeče - TPP str.53

%% makra pro PlainTex?
% \input makra1.tex

%% tohle možná dělá LaTeX automaticky - nebo stačí nějaký balíček 
% \usepglinks 1  % udělá ze stránek rejstříku klikací

%% určitě je pro to v LaTeXu alternativa
% \draft % tiskne na poslední stranu seznam komentářů, co se má dodělat , viz opmac.tex ř. 889
\def\frac#1#2{{#1\over#2}} %definice příkazu \frac z LaTeXu  

%\input prvni_strana
\input zacatek.tex
%\input zac_pojmy
%\input cpp.tex
%\input cpp2.tex
%\input cpppr.tex
%\input elektronika.tex
% \input bity_bajty.tex
%\input software.tex
 

%\chap Ostatní

%\input mechanika.tex
 
\section{Řešení některých problémů}

\subsection{Chyba při uploadu programu do desky Arduino nano:}

% \begtt Please specify 'upload_port' environment or use global '--upload-port' option \endtt

Řešení: pomohlo spustit příkaz {\tt sudo service udev restart} a znovu spustit VS code a odpojit a připojit desku.  

% \input prilohy

% \vfil \break \nonum \sec Rejstřík 

% \begmulti 3 
% \makeindex 
% \endmulti

% \vfil \break \nonum \sec Obsah %\notoc

% \maketoc


% \makerfc

% \bye
 
  
% "-translate-file=cp1250cs" "%Name%.tex"
% \input todo.tex


\printindex % vytiske rejstřík


\tableofcontents % zobrazí obsah

\end{document}

