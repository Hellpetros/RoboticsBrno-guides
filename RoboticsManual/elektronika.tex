\chap Elektronika

\sec Základní součástky

\secc{Rezistor}

{\bf Rezistor} \ii{rezistor} nebo také odpor\fnote{Správně se součástce říká rezistor, odpor je její vlastnost. Ale mnoho lidí používá pro označení součástky slovo odpor.} je součástka, která klade elektrickému proudu  určitý odpor neboli ho omezuje. Toho se používá jako ochrana před zničením čipu nebo jeho části. Odpor se značí $R$. Jednotkou odporu je 1~Ohm, značka $\Omega$. 

Dva rezistory (nebo jiné součástky) mohou být zapojeny buď \dest[ref:serie] {\bf sériově} (tj. za sebou) nebo {\bf paralelně} (tj. vedle sebe), viz obrázek \ref[rezistory]. \ii zapojení/sériové,@ \ii zapojení/paralelní,@      
\midinsert
\picw=0.5\hsize \centerline{\inspic soubory/rezistory.jpg }
\nobreak\medskip
\label[rezistory] \caption/f vlevo: značka rezistoru, uprostřed: rezistory zapojené sériově, vpravo: rezistory zapojené paralelně  
\endinsert

\secc{Kondenzátor}
{\bf Kondenzátor}\ii{kondenzátor} je součástka, která uchovává elektrický náboj. Jeho hlavní vlastností je kapacita. Jednotkou kapacity je Farad, značka F. V~praxi se používají násobky jako mikrofarad ( $\mu$ F), nanofarad (nF) a~pikofarad\fnote{Mikrofarad je miliontina faradu, nanofarad je tisíckrát menší a~pikofarad milionkrát menší než mikrofarad.} (pF). Kondenzátory se nabíjí a~vybíjí různě rychle a~mají různou kapacitu. Keramické kondenzátory 
\ii kondenzátor/keramický,@ mají nejmenší kapacitu(pF, nF) a~jsou nejrychlejší, tantalové \ii kondenzátor/tantalový,@ mívají kapacitu okolo pár $\mu$F a~jsou pomalejší a~nejpomalejší jsou elektrolytické \ii kondenzátor/elektrolytický,@ s~kapacitou stovek až tisíců $\mu$F. U~tantalových a~elektrolytických kondenzátorů musíme dát pozor na polaritu, tj. kam připojujeme + a~kam -. Další důležitý údaj je maximální hodnota napětí, kterou kondenzátor snese. 

\secc{Dioda}

{\bf Dioda}\ii{dioda} je součástka, která usměrňuje elektrický proud. To znamená, že pokud ji zapojíme do elektrického obvodu, tak zajistí, že proud bude téct pouze jedním směrem. Proto budeme diodu používat jako ochranu 
proti tzv. {\bf přepólování} \ii přepolóvání/ochrana -- chybnému zapojení baterie nebo součástky do obvodu, které obvykle vede ke zničení součástky. 
U~samotné diody také záleží na polaritě, tj. při jejím zapojení musíme dávat pozor, kde má kladný pól a~kde záporný. 

Na diodě vzniká úbytek napětí, se kterým musíme počítat při návrhu obvodu. Tak například pokud připojím na diodu s~úbytkem napětí 0,6 V~připojím 12 V, tak za diodou bude napětí 11,4 V. 

Ze začátku nám bude stačit, pokud budeme používat diody   
\link[ref:1N4148]{\Magenta}{1N4148}\Black \
a~\link[ref:1N4007]{\Magenta}{1N4007}\Black .

 

\secc{LED}

{\bf LED}\fnote{Light emitting diode -- světlo vysílající dioda}
\ii{LED} \ii dioda/LED \ii LED \dest[ref:LED] je součástka, která není primárně určená k~usměrnění proudu, ale k~signalizaci, zda obvodem protéká proud. K~LED se vždy musí připojit rezistor. 

\secc{Tranzistor}

{\bf Tranzistor} \ii tranzistor \dest[ref:tranzistor] je součástka, která umožňuje pomocí malých proudů z~čipu řídit větší proudy, například do reproduktoru nebo motorku. 

Tranzistor má tři nožičky: \ii báze {\bf báze}, \ii kolektor {\bf kolektor} a \ii emitor {\bf emitor}. Tranzistorů existuje obrovské množství, pro jednoduchost se budeme zabývat pouze tzv. bipolárními tranzistory. Ty existují ve dvou provedeních PNP a~NPN\fnote{Pro lepší zapamatování značky: eN-Pé-eN - šipka VEN}. Tranzistory mají prakticky dvě použití: mohou pracovat jako spínač (vypínač) nebo jako zesilovač. Budeme se zabývat jednodušším použitím, tj. jako spínače. Budeme používat tranzistory NPN. Pokud bude přes bázi do emitoru téct omezený (malý) proud, tranzistor se otevře a~přes kolektor do emitoru bude téct velký proud. Tak nám stačil malý proud k~řízení velkého. A~toho budeme využívat. 

Ze začátku nám bude stačit používat tranzistory 
\link[ref:BCC337]{\Magenta}{BCC337}\Black , 
\link[ref:BCC547]{\Magenta}{BCC547}\Black \
a~\link[ref:BD911]{\Magenta}{BD911}\Black .


\secc{Cívka}

{\bf Cívka} \ii{cívka} neboli {\bf tlumivka} \ii{tlumivka} je součástka, jejíž hlavní vlastností je indukčnost, jednotka henry, značka H. V~praxi se používají milihenry (mH) a~mikrohenry ( $\mu$H).

\secc{Driver}

Čip nemůže řídit například motor přímo, protože jedním pinem může protékat obvykle maximálně 40~mA. Většina motorů potřebuje mnohem větší proud. Proto se používají součástky zvané \ii{driver} {\bf drivery}, které podle pokynů z~čipu řídí proud z~baterií do motorů a~servomotorů. \dest[ref:drivery] 

\secc{Stabilizátor}

Většinu čipů je potřeba napájet přesně 5~V~nebo 3,3~V. Jak toho docílit z~baterií, na kterých je například 9~V~nebo 12~V, zkrátka více než 5~V? Navíc napětí na bateriích kolísá podle toho, jak moc proudu zrovna odebírají motory. 
Pro napájení čipů je proto nutné použít stabilizátor. 

{\bf Stabilizátor} \ii{stabilizátor} je součástka, která z~kolísavého vyššího napětí vyrobí přesné napětí nižší. %Přitom nějaké napětí také sama spotřebuje. 
Nejčastěji se používají stabilizátory řady 78XX, kde XX značí, na kolik voltů součástka stabilizuje, například 7805 stabilizuje na 5~V. 

Aby stabilizátor mohl pracovat správně, je potřeba, aby napětí, které přivedeme na jeho vstup, bylo aspoň o 2~V~vyšší než které potřebujeme, tj. pokud budu chtít stabilizovat napětí na 5~V, musím stabilizátor napájet aspoň 7~V. Přesné hodnoty pro každý stabilizátor jsou v~jeho datasheetu. 

Stabilizátor má tři piny (vstup, zem, výstup). Zapojí se takto: kladný pól baterie (+) se napojí na vstup, záporný pól na zem (-). Mikrokontrolér se zapojí pinem VCC na výstup stabilizátoru a~GND  se zapojí na zem stabilizátoru. Tímto máme připojený mikrokontrolér na napájení.


Pokud máme stabilizátor před sebou tak, abychom přečetli jeho označení, např. L7805, potom první pin zleva je vstup, druhý je zem a~třetí, tj. úplně vpravo je výstup. Na vstup připojíme 7~V~až 12~V, prostřední pin uzemníme, a~poslední pin vyvedeme na VCC mikrokontroléru. Dále je potřeba věnovat pozornost zapojení kondenzátorů. Mezi vstup a~zem připojím podle datasheetu kondenzátor s~kapacitou 330~nF. Mezi výstup a~zem kondenzátor 100~nF. 



\secc{Krystal}
Pokud budeme potřebovat provozovat některé procesory na vyšší frekvenci, použijeme \ii{krystal} {\bf krystal}. Například původní frekvence mikrokontroléru ATMega16 je nastavena na 1 MHz. S~pomocí krystalu ji můžeme zvýšit až na 16 MHz. 
%Je však důležité si zjistit jakou Atmegu máme, protože na 16 MHz potřebujeme Atmega16-16, existuje však varianta Atmega16-8 a ta je "pouze" na 8 MHz. 
Krystal zapojíme takto: Jeden pin krystalu (je jedno, který) připojíme na pin XTAL1 a~druhý na XTAL2. Dále na pin XTAL1 připojíme jednu nožičku kondenzátoru a~druhou na digitální zem (11-GND). To samé u~pinu XTAL2. Hodnotu kondenzátoru můžeme volit od 12pF do 22pF.

\sec Základní veličiny v~elektronice 

\secc{Proud} 

Pokud \ii proud/elektrický,@ použijeme vodní model, tak (elektrický) proud je množství vody, které proteče vodičem za jednu sekundu.\fnote{Ve skutečnosti je to protečení množství elektronů nebo jiných nosičů el. nábojů s~celkovým nábojem 1 Coulomb za sekundu.} Značka: $I$, jednotka: 1~A~= 1 ampér. 


\secc{Napětí} 
Elektrické napětí \ii elektrické/napětí,@ měříme vždy mezi dvěma body. Můžeme si ho představit jako rozdíl výšek dvou vodních hladin. Z~výše položeného jezera (kladné napětí, +) teče voda (el. proud) do níže položeného (zem, nulový potenciál, -). Značka: $U$, jednotka: 1~V~= 1 volt.


\secc{Odpor} 

Mezi napětím $U$ a~elektrickým proudem $I$ platí vztah:

$$U = R \cdot I,$$ 
Tento vztah je znám pod názvem {\bf Ohmův zákon} \ii zákon/Ohmův,@ . 

Konstanta $R$ se nazývá elektrický odpor, měříme ho ohmech, značka $\Omega$. 

Z~Ohmůva zákona můžeme vyjádřit proud $ I $: 
$$ I = U / R .$$ Ze vzorce je vidět, že pokud použijeme rezistor s~větším odporem při stejném napětí, tak se protékající proud zmenší. 

\secc{Výkon} 
Výkon \ii{výkon} je definován jako součin napětí a~proudu:
$$P = U \cdot I.$$ Značka: $P$, jednotka: 1~W = 1 watt.


\secc{Výpočet tepelného výkonu (ztrátového)} 
Na součástkách, na kterých je úbytek napětí a~kterými protéká proud, vznikají \ii výkon/tepelný,@ tepelné ztráty. 

{\bf Příklad 1:} Na rezistoru je úbytek napětí 3,6~V~a~protéká jím 240~mA. Jaký je tepelný ztrátový výkon?
$$P = U \cdot I = \rm 3,6\ V \cdot 0,24\ A = 0,864\ W = 864\ mW$$
Ztráty na rezistoru činí 864~mW, a~proto musíme volit rezistor, který zvládne větší výkon, tj. minimálně 1~W. 

{\bf Příklad 2:} Diodou 1N4148 bude procházet 100~mA, při tomto proudu bude úbytek napětí na diodě 1~V. Nezničíme diodu 1N4148?
$$P = U \cdot I = \rm 1\ V \cdot 0,1\ A = 0,1\ W = 100\ mW$$
Diodu nezničíme, protože je vyráběná na celkový výkon 500~mW. 

{\bf Příklad 3:} Na stabilizátor L7805 přivádím 12~V, stabilizátor mi vytváří 5~V~stabilizovaného napětí a~odebírám z~něho 250~mA. Jaký tepelný výkon bude potřeba uchladit?
$$P = U \cdot I = \rm (12\ V - 5\ V) \cdot 0,25\ A = 1,75\ W$$
Tepelné ztráty na stabilizátoru budou 1,75~W\fnote{Ve vzorci bylo použito 12~V~-- 5~V, je to protože na stabilizátoru ubyde 7~V, abychom se dostali na požadovaných 5~V.}.

%%%%%%%%%%%%%%%%%%%%%%%%%%%%%%%%%%%%%%%%%%%%%%%%%%%%%%%%%%%%%%%%%

\secc Výpočet rezistoru pro LED 

LED \ii dioda/LED,@ \ii rezistor \ii LED je součástka, která není primárně určená k~usměrnění proudu, ale k~signalizaci. Může svítit světlem bílým, modrým, zeleným, červeným, ultrafialovým či infračerveným\fnote{Toho využijeme jako senzoru pro robota.}.

Pokud připojím diodu správně na napětí, tj. tak aby mohl procházet proud a~ono přesto nic, tak jsem diodu spálil, protože jí tekl moc velký proud. A~proto musíme vždy k~diodě připojit do  \link[ref:serie]{\Magenta}{série} \Black 
 rezistor, který omezí proud protékající přes LED. A~to podle vzorce:

$$R = \frac{U}{I} = \rm \frac{Napeti\_zdroje - Ubytek\_napeti}{Pozadovany\_proud}$$

{\bf Příklad 1:} Vypočítejte odpor rezistoru, který zapojíme do série k~LED. Připojujeme k~ní napětí 5~V~a~provozní proud je 20~mA a~maximální proud, při kterém dojde ke zničení diody je 40~mA. Úbytek na diodě je 1,2~V.
$$R = \frac{U}{I} = \rm \frac{5\ V - 1,2 \ V}{0,02 \ A} = 190\ \Omega$$
Použijeme rezistor s~odporem $190\ \Omega$ nebo nejbližší vyšší odpor.

{\bf Příklad 2:} Máme sériově spojeny tři LED, s~úbytky napětí 0,6~V, 0,8~V~a~1,2~V. Připojíme je k~zdroji o napětí 12~V. Rezistor o jak velkém odporu musíme použít, jestliže má diodami protékat 20~mA?

$$R = \frac{U}{I} = \rm \frac{12\ V-0,6\ V-0,8\ V-1,2\ V}{0,02 \ A} = 470\ \Omega$$

Musíme použít rezistor o odporu $470\ \Omega$. 

Měli bychom ještě spočítat tepelný výkon rezistoru. Můžeme použít úbytek napětí na rezistoru vynásobený procházejícím proudem:
$$P = U \cdot I = \rm (12\ V-0,6\ V-0,8\ V-1,2\ V) \cdot 0,02\ A = 9,4\ V \cdot 0,02\ A = 0,188\ W = 188\ mW$$
Stačí 250~mW rezistor. Nebo si můžu vypočítat úbytek napětí z~$U = R \cdot I$, to protože znám odpor a~protékající proud.

$$P = U \cdot I = R \cdot I \cdot I = R \cdot I^2 = 470\ \Omega \cdot \rm (0,02\ A)^2 = 0,188\ W = 188\ mW$$

Vyšel nám stejný výsledek, použijeme tedy 250~mW rezistor o odporu $470\ \Omega$.


%%%%%%%%%%%%%%%%%%%%%%%%%%%%%%%%%%%%%%%%%%%%%%%%%%%%%%%%%%%%%%%%%
\sec Motorky, serva a~PWM 

\secc Motorky

Motorek připojujeme vždy přes \link[ref:tranzistor]{\Magenta}{tranzistor}\Black .  
Pokud chceme řídit motorek z čipu stylem start -- stop, postačí přes odpor např. 1~k$\Omega$ spojit výstupní pin čipu s bází tranzistoru a na emitor a kolektor připojit baterii a motorek zapojený do série. 

Dále je potřeba bázi tranzistoru propojit pomocí např. 10~k$\Omega$ se zemí. Jinak totiž při vypnutém signálu z čipu báze \uv{visí v luftě} a chová se jako anténa -- indukují se na ní různé signály a většinou se díky tomu motorek samovolně slabě otáčí.   

Nakonec je potřeba mít společnou zem pro čip i pro motorek -- pokud to není splněno, obvykle motorek nejede. 

Pro větší motorky potřebujeme lepší tranzitor nebo dva tranzistory v tzv. Darlingtonově zapojení (viz web). Další možností jsou speciální integrované obvody pro řízení motorů, tzv.  \link[ref:drivery]{\Magenta}{drivery}\Black .  

Pokud chceme řídit motory programově pomocí PWM (viz další kapitola), musíme navíc mezi emitor a kolektor tranzistoru vložit (obyčejnou) diodu pólovanou závěrně vůči baterii. Při vypnutí tranzistoru vznikají totiž na cívkách motorku napěťové špičky, které deformují tvar PWM signálu.  Další proudy se indukují, když se motorek po vypnutí proudu setrvačností otáčí dál. 

\secc PWM
\dest[ref:PWM] \iid PWM \rfc{pwm} 
doplnit 

\secc Řízení serva

\ii servo Jak se řídí pohyb serva? Pro tento účel je ideální právě generování PWM signálu. 

Servo se řídí logickým signálem (jedničkou) po dobu od 1~ms do 2~ms (často i~od 0,5~ms do 2,5~ms), a~celková perioda je 20~ms. Podle toho, jak dlouho signál trvá, tak se servo natočí. Tj. pokud budeme chtít servo maximálně natočit na jednu stranu, nastavíme pin, který slouží jako řídící signál pro servo, na logickou jedničku po dobu 1~ms a~pak 19~ms logickou nulu a~pak zase logickou jedničku, logickou nulu, atd... 

Pokud budeme chtít servo posunout do druhé krajní polohy, necháme logickou jedničku po dobu 2~ms a~logickou nulu po dobu 18~ms. Pokud budeme chtít střední polohu, tak jedničku nastavíme na 1,5~ms a~nulu na 18,5~ms. Jestliže budeme potřebovat jiný úhel natočení, nastavíme logickou jedničku na odpovídající dobu.

{\bf Pozor!} Servo nesnáší přepólování napětí, když se přepóluje, tak shoří (když se přepóluje signál, tak to tolik nevadí).

\sec ALKS a~další desky 

\secc ESP 32  

Deska \ii ESP~32/deska,@ ESP~32 je vývojová deska osazená čipem {\tt ESP-WROOM-32}, který má řadu výborných vlastností.\fnote{url{http://navody.arduino-shop.cz/navody-k-produktum/vyvojova-deska-esp32.html} }

Deska se napájí z~\iid USB (5~V) a~je na ní napěťový převodník\rfc{? napěťový převodník} na 3,3~V. Přitom USB může dodávat oficiálně 100~mA, v~reálu ale běžně dodává 500~mA až 1 A. USB porty jsou také vcelku odolné proti zkratu.   

\secc ALKS

Deska ALKS\fnote{Arduion Learning Kit Starter} byla navržena přímo na Robotárně Brno právě proto, že 
hotová \ii ALKS/deska,@ deska   vás hlavně ze začátku zbavuje nutnosti vědět, co si můžete dovolit kam připojit a~jestli to bude fungovat. 

Na ALKS se dají nasadit desky ESP~32, Arduino uno  a~Arduino nano, které jí také poskytují napájení.  

Stránky  věnované desce ALKS jsou zde: \hfil \break  \url{https://github.com/RoboticsBrno/ArduinoLearningKitStarter}  

Zapojení \emergencystretch=2cm desky ALKS je zde: \url{https://github.com/RoboticsBrno/ArduinoLearningKitStarter/blob/master/docs/ArduinoLearningKitStarter.pdf}

Zapojení pinů desky ALKS zde: \url{https://github.com/RoboticsBrno/ArduinoLearningKitStarter/blob/master/docs/pinout.pdf}

Při připojování čehokoliv dalšího k~této nebo jiné desce si nechte před zapojením napájení všechno zkontrolovat. Hlavně, pokud to žere víc proudu (serva a~motory) nebo pokud to vyžaduje vyšší napětí. 

