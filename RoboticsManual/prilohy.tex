\nonum \sec Příloha A~-- Hodnoty vybraných součástek

{\bf Dioda 1N4148}  \dest[ref:1N4148] \ii dioda/1N4148,@ 

Maximální napětí: 100 V

Maximální proud: 200 mA

Maximální výkon:  500 mW

Úbytek napětí: 1,8 V

\vskip 2mm

{\bf Dioda 1N4007} \dest[ref:1N4007] \ii dioda/1N4007,@ 

Maximální napětí: 1000 V

Maximální proud: 1 A

Maximální výkon: 3 W

Úbytek napětí: 1,1 V

\vskip 2mm

{\bf tranzistor BC337} \dest[ref:BCC337] \ii tranzistor/BCC337,@

Max. napětí mezi kol. a~emit. $V \rm _{CEO}$:50 V

Max. napětí mezi bází a~emit. $V \rm _{CBO}$: 5 V

Max. proud tekoucí kolektorem $I \rm _C$: 800 mA

Maximální výkon $ P \rm _C$: 625 mW

Zesílení $ h \rm _{fe}$: 100 až 600

\vskip 2mm

{\bf tranzistor BC547} \dest[ref:BCC547] \ii tranzistor/BCC547,@

Max. napětí mezi kol. a~emit. $ V \rm_{CEO}$: 45 V

Max. napětí mezi bází a~emit. $ V \rm _{CBO}$: 6 V
 
Max. proud tekoucí kolektorem $ I \rm _C$: 100 mA

Maximální výkon $P \rm _C$: 500 mW

Zesílení $h \rm _{fe}$: 110 až 800

\vskip 2mm

{\bf tranzistor BD911}  \dest[ref:BD911] \ii tranzistor/BD911,@

Max. napětí mezi kol. a~emit.$V \rm _{CEO}$: 100 V

Max. napětí mezi bází a~emit. $V \rm _{CBO}$: 5 V

Max. proud tekoucí kolektorem $I \rm _C$: 15 A

Maximální výkon $P \rm _C$: 90 W

Zesílení $h \rm _{fe}$: 5 až 250

Důležité je si všimnout, že minimální zesílení je 5\fnote{Platí pokud bude kolektorem protékat 10~A.}. To znamená, že pokud budeme chtít tranzistorem spínat proud 10~A, tak budeme muset nechat bází téct řídící proud 2~A, to znamená, že ho nemůžeme naplno využít, pokud ho budeme řídit mikrokontrolérem, tj. musíme ho spínat jiným tranzistorem. Anebo přijdeme na myšlenku, že pokud budeme chtít řídit velké proudy, budeme potřebovat tranzistory typu MOS-FET, např.: IRF520, IRL3803.

%Maximální proud tekoucí kolektorem $I_C$\footnote{Ang.: Collector Current}

%Maximální výkon $P_C$\footnote{Ang.: Collector Power Dissipation}

%Zesílení $h_{fe}$\footnote{Ang.: DC Current Gain}

\nonum \sec Příloha B -- Poznámky a~vize

 \secc Linux

Drivery do tiskáren mají příponu ppd. Napřed se musí nainstalovat program cups. 

Irfan view jede i~v~Linuxu\fnote{\url{http://www.boekhoff.info/install-irfan-view-on-linux/}}.

%You can also use the Screenshot program included in Ubuntu (hit the Super key and type "Screenshot") which will afterwards ask you where you want to save the screenshot. If you just want a quick whole-screen shot you can hit the Printscreen key.


 \secc Bluetooth

\dest[ref:bluetooth] Moduly bluetooth slouží ke komunikaci mezi dvěma čipy nebo počítačem a čipem (nebo jiným zařízením). Mohou být napájeny 5V nebo 3,3V. 
 \rfc{bluetooth}

Propojujeme vždy pin Rx na jednom čipu s pinem Tx na druhém čipu. 

\secc Připojení k počítači pomocí bluetooth

Nový bluetooth (zub) se musí napoprvé vyhledat a aktivovat v počítači. Pokaždé se musí připojit a zkontrolovat - když je komunikace v pořádku (aktivována, ale nemusí se přenášet data ), svítí LED na zubu. Když je v pořádku modul v počítači, tak bliká.  

Dále musí být spojená země zubu a čipu 

Kód zubů: 007 



\secc Přerušení

Co je přerušení? Procesor může zvládnout pouze jednu operaci na jeden tik krystalu, postupuje od jednoho příkazu k~druhému a~nemůže jen tak všeho nechat a~věnovat se něčemu jinému, občas je to ale potřeba. Při přerušení procesor všeho nechá a~bude se věnovat přerušení, potom co skončí se bude věnovat dál programu tam, kde přestal. 

\secc Baterie 

Dobíjecí: napětí 1,2 V~\ii baterie/dobíjecí,@

Jednorázové: napětí 1,5 V~\ii baterie/jednorázové,@ 
 
\secc Vize -- co přidat do textu 

Github

Lorris

Adruino IDE 

Laser ve Lablabu 

Osciloskop

Baterie LiPol a~jejich nabíjení 

Převodník napěťových úrovní 

Pájení 

Sběrnice - USART/UART, IIC, SPI, další?
